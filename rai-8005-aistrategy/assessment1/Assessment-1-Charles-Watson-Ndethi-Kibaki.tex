% Options for packages loaded elsewhere
\PassOptionsToPackage{unicode}{hyperref}
\PassOptionsToPackage{hyphens}{url}
\PassOptionsToPackage{dvipsnames,svgnames,x11names}{xcolor}
%
\documentclass[
]{article}

\usepackage{amsmath,amssymb}
\usepackage{iftex}
\ifPDFTeX
  \usepackage[T1]{fontenc}
  \usepackage[utf8]{inputenc}
  \usepackage{textcomp} % provide euro and other symbols
\else % if luatex or xetex
  \usepackage{unicode-math}
  \defaultfontfeatures{Scale=MatchLowercase}
  \defaultfontfeatures[\rmfamily]{Ligatures=TeX,Scale=1}
\fi
\usepackage{lmodern}
\ifPDFTeX\else  
    % xetex/luatex font selection
\fi
% Use upquote if available, for straight quotes in verbatim environments
\IfFileExists{upquote.sty}{\usepackage{upquote}}{}
\IfFileExists{microtype.sty}{% use microtype if available
  \usepackage[]{microtype}
  \UseMicrotypeSet[protrusion]{basicmath} % disable protrusion for tt fonts
}{}
\makeatletter
\@ifundefined{KOMAClassName}{% if non-KOMA class
  \IfFileExists{parskip.sty}{%
    \usepackage{parskip}
  }{% else
    \setlength{\parindent}{0pt}
    \setlength{\parskip}{6pt plus 2pt minus 1pt}}
}{% if KOMA class
  \KOMAoptions{parskip=half}}
\makeatother
\usepackage{xcolor}
\usepackage[margin=1in]{geometry}
\setlength{\emergencystretch}{3em} % prevent overfull lines
\setcounter{secnumdepth}{5}
% Make \paragraph and \subparagraph free-standing
\makeatletter
\ifx\paragraph\undefined\else
  \let\oldparagraph\paragraph
  \renewcommand{\paragraph}{
    \@ifstar
      \xxxParagraphStar
      \xxxParagraphNoStar
  }
  \newcommand{\xxxParagraphStar}[1]{\oldparagraph*{#1}\mbox{}}
  \newcommand{\xxxParagraphNoStar}[1]{\oldparagraph{#1}\mbox{}}
\fi
\ifx\subparagraph\undefined\else
  \let\oldsubparagraph\subparagraph
  \renewcommand{\subparagraph}{
    \@ifstar
      \xxxSubParagraphStar
      \xxxSubParagraphNoStar
  }
  \newcommand{\xxxSubParagraphStar}[1]{\oldsubparagraph*{#1}\mbox{}}
  \newcommand{\xxxSubParagraphNoStar}[1]{\oldsubparagraph{#1}\mbox{}}
\fi
\makeatother


\providecommand{\tightlist}{%
  \setlength{\itemsep}{0pt}\setlength{\parskip}{0pt}}\usepackage{longtable,booktabs,array}
\usepackage{calc} % for calculating minipage widths
% Correct order of tables after \paragraph or \subparagraph
\usepackage{etoolbox}
\makeatletter
\patchcmd\longtable{\par}{\if@noskipsec\mbox{}\fi\par}{}{}
\makeatother
% Allow footnotes in longtable head/foot
\IfFileExists{footnotehyper.sty}{\usepackage{footnotehyper}}{\usepackage{footnote}}
\makesavenoteenv{longtable}
\usepackage{graphicx}
\makeatletter
\newsavebox\pandoc@box
\newcommand*\pandocbounded[1]{% scales image to fit in text height/width
  \sbox\pandoc@box{#1}%
  \Gscale@div\@tempa{\textheight}{\dimexpr\ht\pandoc@box+\dp\pandoc@box\relax}%
  \Gscale@div\@tempb{\linewidth}{\wd\pandoc@box}%
  \ifdim\@tempb\p@<\@tempa\p@\let\@tempa\@tempb\fi% select the smaller of both
  \ifdim\@tempa\p@<\p@\scalebox{\@tempa}{\usebox\pandoc@box}%
  \else\usebox{\pandoc@box}%
  \fi%
}
% Set default figure placement to htbp
\def\fps@figure{htbp}
\makeatother
% definitions for citeproc citations
\NewDocumentCommand\citeproctext{}{}
\NewDocumentCommand\citeproc{mm}{%
  \begingroup\def\citeproctext{#2}\cite{#1}\endgroup}
\makeatletter
 % allow citations to break across lines
 \let\@cite@ofmt\@firstofone
 % avoid brackets around text for \cite:
 \def\@biblabel#1{}
 \def\@cite#1#2{{#1\if@tempswa , #2\fi}}
\makeatother
\newlength{\cslhangindent}
\setlength{\cslhangindent}{1.5em}
\newlength{\csllabelwidth}
\setlength{\csllabelwidth}{3em}
\newenvironment{CSLReferences}[2] % #1 hanging-indent, #2 entry-spacing
 {\begin{list}{}{%
  \setlength{\itemindent}{0pt}
  \setlength{\leftmargin}{0pt}
  \setlength{\parsep}{0pt}
  % turn on hanging indent if param 1 is 1
  \ifodd #1
   \setlength{\leftmargin}{\cslhangindent}
   \setlength{\itemindent}{-1\cslhangindent}
  \fi
  % set entry spacing
  \setlength{\itemsep}{#2\baselineskip}}}
 {\end{list}}
\usepackage{calc}
\newcommand{\CSLBlock}[1]{\hfill\break\parbox[t]{\linewidth}{\strut\ignorespaces#1\strut}}
\newcommand{\CSLLeftMargin}[1]{\parbox[t]{\csllabelwidth}{\strut#1\strut}}
\newcommand{\CSLRightInline}[1]{\parbox[t]{\linewidth - \csllabelwidth}{\strut#1\strut}}
\newcommand{\CSLIndent}[1]{\hspace{\cslhangindent}#1}

\usepackage{microtype}
\usepackage{setspace}
\onehalfspacing
\renewcommand{\familydefault}{\sfdefault}
\usepackage{fontspec}
\setmainfont{/mnt/c/windows/fonts/ARIALN.TTF}
\sloppy
\setlength{\emergencystretch}{3em}
\usepackage{etoolbox}
\AtBeginEnvironment{quote}{\small\ttfamily}
\makeatletter
\@ifpackageloaded{caption}{}{\usepackage{caption}}
\AtBeginDocument{%
\ifdefined\contentsname
  \renewcommand*\contentsname{Table of contents}
\else
  \newcommand\contentsname{Table of contents}
\fi
\ifdefined\listfigurename
  \renewcommand*\listfigurename{List of Figures}
\else
  \newcommand\listfigurename{List of Figures}
\fi
\ifdefined\listtablename
  \renewcommand*\listtablename{List of Tables}
\else
  \newcommand\listtablename{List of Tables}
\fi
\ifdefined\figurename
  \renewcommand*\figurename{Figure}
\else
  \newcommand\figurename{Figure}
\fi
\ifdefined\tablename
  \renewcommand*\tablename{Table}
\else
  \newcommand\tablename{Table}
\fi
}
\@ifpackageloaded{float}{}{\usepackage{float}}
\floatstyle{ruled}
\@ifundefined{c@chapter}{\newfloat{codelisting}{h}{lop}}{\newfloat{codelisting}{h}{lop}[chapter]}
\floatname{codelisting}{Listing}
\newcommand*\listoflistings{\listof{codelisting}{List of Listings}}
\makeatother
\makeatletter
\makeatother
\makeatletter
\@ifpackageloaded{caption}{}{\usepackage{caption}}
\@ifpackageloaded{subcaption}{}{\usepackage{subcaption}}
\makeatother

\usepackage{bookmark}

\IfFileExists{xurl.sty}{\usepackage{xurl}}{} % add URL line breaks if available
\urlstyle{same} % disable monospaced font for URLs
\hypersetup{
  pdftitle={AI Strategy for Pharo Foundation Finance Transformation},
  pdfauthor={Charles Watson Ndethi Kibaki},
  pdfkeywords={AI strategy, finance transformation, Pharo
Foundation, artificial intelligence, financial
management, non-profit, for-profit},
  colorlinks=true,
  linkcolor={blue},
  filecolor={Maroon},
  citecolor={Blue},
  urlcolor={Blue},
  pdfcreator={LaTeX via pandoc}}


\title{AI Strategy for Pharo Foundation Finance Transformation}
\author{Charles Watson Ndethi Kibaki}
\date{March 23, 2025}

\begin{document}
\maketitle
\begin{abstract}
This report outlines an AI strategy for Pharo Foundation's finance
transformation, focusing on enhancing financial management and
operational efficiency across Pharo Development and Pharo Ventures. The
strategy employs the 3Rs framework---Replace, Reimagine, and
Recombine---to integrate AI into finance functions. Key initiatives
include a unified financial intelligence platform, intelligent process
automation, predictive analytics, compliance and risk management tools,
and a financial decision support system. The implementation roadmap
spans 36 months, addressing data quality, connectivity, and skill gaps.
Successful adoption of this strategy will improve operational
efficiency, collaboration, financial controls, and decision-making,
aligning with Pharo's mission of fostering self-reliance in Africa.
\end{abstract}


\section{AI Strategy for Pharo Foundation Finance
Transformation}\label{ai-strategy-for-pharo-foundation-finance-transformation}

\subsection{Executive Summary}\label{executive-summary}

The Pharo Foundation, encompassing both Pharo Development (non-profit)
and Pharo Ventures (for-profit), is dedicated to fostering
self-sufficiency in Africa through education, water management, and
productivity enhancement. As the organization advances its finance
transformation strategy to enhance financial management and operational
efficiency, artificial intelligence (AI) presents significant
opportunities to accelerate and deepen this transformation.

This AI strategy proposes a structured approach to integrating AI
capabilities into Pharo Foundation's finance functions, with an emphasis
on the 3Rs framework: Replace manual processes, Reimagine operations,
and Recombine existing systems. Key initiatives include implementing a
unified financial intelligence platform, intelligent process automation,
predictive analytics, compliance and risk management tools, and a
financial decision support system.

By implementing this AI strategy, Pharo Foundation can expect enhanced
operational efficiency across both Development and Ventures arms,
improved collaboration across countries of operation, strengthened
financial controls, and enhanced decision-making capabilities.
Implementation will follow a three-phase approach over 36 months,
requiring investments in technology infrastructure, talent development,
and organizational change management.

For successful implementation, the Foundation must address data quality
challenges, connectivity issues in African regions, and skill gaps while
ensuring AI applications align with the organization's mission of
fostering self-reliance in Africa. With thoughtful implementation, AI
can become a powerful enabler of Pharo Foundation's vision of ``a
vibrant, productive, and self-reliant Africa.''

\subsection{Introduction}\label{introduction}

Pharo Foundation operates through two main arms: Pharo Development,
focusing on public goods and not-for-profit projects, and Pharo
Ventures, investing in private sector and for-profit ventures. Both arms
work toward achieving Pharo's missions but utilize different tools and
approaches (Pharo Foundation, 2023). This dual structure creates unique
financial management challenges that must be addressed through a
comprehensive transformation strategy.

The Foundation has recently embarked on a finance transformation
journey, moving from Xero and Approvalmax to Microsoft Dynamics Business
Central (BC) as its enterprise resource planning (ERP) system. This
migration aims to modernize financial systems, streamline financial
processes, and enhance financial data management to improve
decision-making and operational efficiency (Pharo Foundation Finance
Transformation Strategy, 2024). However, implementing BC has faced
challenges, including user adaptation issues and inconsistent workflows
across countries of operation.

Artificial intelligence presents a significant opportunity to enhance
this finance transformation. As Castelo et al.~(2019) note, ``AI is on
the verge of reshaping the future of business operations, with the
market for AI in transportation projected to reach \$10.3 billion by
2030.'' In finance specifically, AI can automate routine tasks, enhance
decision-making with data-driven insights, improve compliance
monitoring, and optimize resource allocation (Jung \& Seiter, 2021).

This AI strategy outlines how Pharo Foundation can leverage AI
technologies to accelerate and enhance its finance transformation,
supporting both arms in achieving their missions more effectively. The
strategy aligns with Pharo's vision of ``a vibrant, productive, and
self-reliant Africa'' by enabling more efficient, transparent, and
data-driven financial operations that will ultimately strengthen the
Foundation's impact across its education, water, and productivity
missions.

\subsection{AI Strategy}\label{ai-strategy}

\subsubsection{Strategic Business
Objectives}\label{strategic-business-objectives}

The AI strategy for Pharo Foundation's finance transformation is
designed to support the organization's vision of ``a vibrant,
productive, and self-reliant Africa'' by enhancing financial operations
across both Pharo Development and Pharo Ventures. This strategy directly
supports Pharo's three core missions:

\begin{enumerate}
\def\labelenumi{\arabic{enumi}.}
\item
  \textbf{Education Mission}: By optimizing financial resources and
  improving resource allocation decisions, AI will help the Foundation
  maximize the impact of its educational initiatives, from early
  childhood education to vocational training.
\item
  \textbf{Water Mission}: AI-enhanced financial planning will improve
  cost-effectiveness and sustainability of water infrastructure
  projects, ensuring communities across Africa have access to safe and
  affordable water sources.
\item
  \textbf{Productivity Mission}: Through better financial analytics and
  decision support, the Foundation can more effectively eliminate
  barriers to employment and productivity for working people.
\end{enumerate}

At the operational level, the AI strategy aims to enhance the finance
transformation by: - Modernizing financial systems through intelligent
automation and advanced analytics - Streamlining financial processes
through AI-powered workflow optimization - Enhancing financial data
management with improved data integration and quality controls -
Improving decision-making through predictive analytics and scenario
modeling - Maintaining strong controls and visibility through
AI-enhanced risk management

As Brynjolfsson and Mitchell (2017) observe, ``AI is driven by data. To
fully leverage the value of AI at an enterprise level, organizations
should focus on their proprietary information and treat it as a
high-value, critical asset.'' For Pharo Foundation, this means
leveraging its unique financial and operational data across multiple
countries to generate insights that drive mission impact.

\subsubsection{Strategic AI Frameworks Application: The 3Rs
Model}\label{strategic-ai-frameworks-application-the-3rs-model}

To structure the AI transformation of Pharo Foundation's finance
function, we apply the 3Rs model: Replace, Reimagine, and Recombine
(Iansiti \& Lakhani, 2020). This framework provides a systematic
approach to identify opportunities for AI implementation across the
finance function.

\paragraph{Replace: Automating Manual Financial
Processes}\label{replace-automating-manual-financial-processes}

The first dimension of transformation involves identifying routine,
manual processes that can be automated through AI:

\begin{itemize}
\item
  \textbf{Financial Document Processing}: AI can extract data from
  invoices, receipts, and financial documents through optical character
  recognition (OCR) and natural language processing (NLP), eliminating
  manual data entry. This is particularly valuable for Pharo
  Foundation's operations across multiple African countries, where
  document standards may vary.
\item
  \textbf{Transaction Categorization}: ML algorithms can automatically
  categorize financial transactions based on historical patterns,
  reducing manual classification work and improving consistency. This
  addresses the current challenge of maintaining consistent financial
  records across Development and Ventures arms.
\item
  \textbf{Basic Reporting and Reconciliation}: AI can automate the
  generation of standard financial reports and perform preliminary
  reconciliation of accounts, freeing finance staff to focus on analysis
  and strategic tasks. This directly supports the objective to
  ``streamline and standardize processes to eliminate bottlenecks,
  improve efficiency and maximize controls'' (Pharo Foundation Finance
  Transformation Strategy, 2024).
\end{itemize}

\paragraph{Reimagine: Transforming Financial
Operations}\label{reimagine-transforming-financial-operations}

The second dimension involves fundamentally rethinking how financial
work is performed:

\begin{itemize}
\item
  \textbf{Proactive Financial Risk Management}: Rather than reactive
  monitoring, AI can continuously analyze financial data to identify
  potential risks before they materialize, enabling proactive
  mitigation. As noted by Maedche et al.~(2019), ``AI-powered risk
  management can transform how organizations identify, assess, and
  mitigate financial risks.''
\item
  \textbf{Intelligent Financial Forecasting}: Moving beyond traditional
  forecasting methods, AI can analyze multiple internal and external
  data sources to generate more accurate predictions about financial
  performance and resource requirements. This addresses the Foundation's
  need for ``data-driven decision-making'' and ``utilizing financial
  data analytics for insights and informed decisions'' (Pharo Foundation
  Finance Transformation Strategy, 2024).
\item
  \textbf{Cross-Entity Financial Insights}: AI can break down silos
  between Pharo Development and Pharo Ventures, identifying patterns and
  relationships that would be difficult to discover manually. This
  directly supports the objective to develop ``a unified financial
  system that integrates data from both non-profit and for-profit
  operations'' (Pharo Foundation Finance Transformation Strategy, 2024).
\end{itemize}

\paragraph{Recombine: Enhancing Existing
Systems}\label{recombine-enhancing-existing-systems}

The third dimension focuses on integrating AI with existing systems to
enhance their capabilities:

\begin{itemize}
\item
  \textbf{AI-Enhanced Cloud Infrastructure}: Adding AI capabilities to
  the Foundation's cloud-based systems (Microsoft 365 and Google Suite)
  can improve data management, security, and accessibility. This aligns
  with the infrastructure modernization goal to ``upgrade financial IT
  infrastructure to cloud-based systems for scalability and
  accessibility'' (Pharo Foundation Finance Transformation Strategy,
  2024).
\item
  \textbf{Intelligent Connectors Between Systems}: AI can facilitate
  seamless data flow between the Foundation's various systems, including
  the new Microsoft Dynamics Business Central ERP, HRIS, EMIS, MES, and
  LMS/HIMS. As Collins et al.~(2021) note, ``AI can significantly
  enhance system integration by intelligently mapping data across
  disparate systems.''
\item
  \textbf{Data-Driven Process Orchestration}: AI can optimize end-to-end
  financial processes by analyzing process performance data and
  recommending improvements, supporting the Foundation's goal of
  ``continuous improvement'' with ``regular reviews and refinements''
  (Pharo Foundation Finance Transformation Strategy, 2024).
\end{itemize}

\subsubsection{Specific AI Use Cases for Pharo's Finance
Transformation}\label{specific-ai-use-cases-for-pharos-finance-transformation}

Based on the 3Rs analysis and aligned with Pharo Foundation's finance
transformation goals, five key AI use cases have been identified:

\paragraph{1. Unified Financial Intelligence
Platform}\label{unified-financial-intelligence-platform}

This platform will serve as a centralized hub for financial data and
insights across Pharo Development and Pharo Ventures:

\begin{itemize}
\item
  \textbf{AI-Powered Data Integration}: Intelligent data integration
  tools will consolidate financial information from multiple systems and
  countries, creating a single source of truth for financial data.
\item
  \textbf{Cross-Entity Financial Analysis}: AI algorithms will analyze
  patterns and relationships across both arms of the Foundation, helping
  leaders understand the full financial picture and identify
  opportunities for synergy.
\item
  \textbf{Automated Multi-Entity Reporting}: The platform will generate
  customized reports for different stakeholder groups, from Board
  members to country managers, with insights relevant to their specific
  needs and responsibilities.
\end{itemize}

\textbf{Benefits}: This platform will provide a consolidated view of
financial activities, improve reporting quality and timeliness, and
enable better cross-entity decision-making. According to Mikalef and
Gupta (2021), ``Organizations with integrated AI-powered financial
systems can achieve up to 25\% improvement in decision-making speed and
accuracy.''

\paragraph{2. Intelligent Financial Process
Automation}\label{intelligent-financial-process-automation}

This use case focuses on automating routine financial tasks through AI:

\begin{itemize}
\item
  \textbf{AI Document Processing}: Advanced OCR and NLP will extract and
  validate data from financial documents, reducing manual data entry and
  errors. This is particularly valuable for operations in countries with
  limited digital infrastructure.
\item
  \textbf{Smart Workflow Routing}: AI will route financial approvals and
  documents based on content, urgency, and organizational rules,
  streamlining processes and reducing bottlenecks.
\item
  \textbf{Exception Detection and Handling}: AI will identify unusual
  transactions or patterns that require human attention, focusing staff
  time on high-value activities while routine transactions are processed
  automatically.
\end{itemize}

\textbf{Benefits}: This automation will reduce manual effort, improve
accuracy, accelerate processing times, and enhance controls. Research by
Davenport and Ronanki (2018) found that ``finance process automation
using AI can reduce processing times by 50-70\% while improving accuracy
by 30-40\%.''

\paragraph{3. Predictive Financial
Analytics}\label{predictive-financial-analytics}

This use case leverages AI to improve financial planning and resource
allocation:

\begin{itemize}
\item
  \textbf{AI-Based Forecasting}: Machine learning models will generate
  accurate forecasts for both non-profit program expenses and for-profit
  venture performance, improving budget planning.
\item
  \textbf{Cash Flow Optimization}: AI algorithms will analyze cash flow
  patterns to optimize liquidity management across the Foundation's
  operations in multiple countries and currencies.
\item
  \textbf{Resource Allocation Recommendations}: AI will analyze program
  performance data to recommend optimal resource allocation to maximize
  impact across the Foundation's three missions.
\end{itemize}

\textbf{Benefits}: These capabilities will enhance financial planning
accuracy, improve resource utilization, and reduce financial risks. As
noted by McKinsey (2024), ``Organizations that implement AI-driven
forecasting can reduce forecast errors by 30-50\% compared to
traditional methods.''

\paragraph{4. AI-Enhanced Compliance and Risk
Management}\label{ai-enhanced-compliance-and-risk-management}

This use case strengthens governance and risk management:

\begin{itemize}
\item
  \textbf{Automated Compliance Monitoring}: AI will continuously monitor
  financial transactions and processes for compliance with relevant
  regulations across the multiple African jurisdictions where Pharo
  operates.
\item
  \textbf{Fraud Detection and Prevention}: Machine learning models will
  identify potential fraudulent activities by detecting unusual patterns
  in financial data, protecting the Foundation's resources.
\item
  \textbf{Risk Assessment for Projects and Ventures}: AI will evaluate
  the financial risks associated with development projects and venture
  investments, supporting better decision-making.
\end{itemize}

\textbf{Benefits}: These capabilities will strengthen governance,
improve risk management, and reduce compliance issues. According to Shin
et al.~(2022), ``AI-powered compliance monitoring can detect up to 90\%
of potential compliance issues before they become problematic.''

\paragraph{5. Financial Decision Support
System}\label{financial-decision-support-system}

This use case enhances decision-making through AI-powered insights:

\begin{itemize}
\item
  \textbf{Scenario Modeling}: AI will generate and evaluate multiple
  scenarios for financial decisions, helping leaders understand
  potential outcomes and tradeoffs.
\item
  \textbf{Impact Analysis}: The system will assess the financial impact
  of decisions on the Foundation's overall mission and specific
  programs, ensuring alignment with strategic goals.
\item
  \textbf{Natural Language Interface}: A conversational AI interface
  will allow non-technical users to query financial data and receive
  insights in natural language, democratizing access to financial
  intelligence.
\end{itemize}

\textbf{Benefits}: This system will enable data-driven decisions,
improve investment outcomes, and make financial intelligence more
accessible across the organization. Research by Lee and See (2004)
indicates that ``AI-powered decision support systems can improve
decision quality by 20-30\% by providing relevant information at the
right time.''

\subsection{Implementation Roadmap}\label{implementation-roadmap}

The implementation of AI for Pharo Foundation's finance transformation
will follow a phased approach over 36 months, balancing quick wins with
long-term capability building. This roadmap considers the Foundation's
current ERP implementation, which has faced some challenges but is
``finally off to the races'' (Pharo Finance Notes, 2024).

\subsubsection{Phase 1: Foundation (0-6
months)}\label{phase-1-foundation-0-6-months}

The initial phase will focus on establishing the necessary
infrastructure, skills, and preliminary use cases:

\paragraph{AI Readiness Assessment}\label{ai-readiness-assessment}

\begin{itemize}
\tightlist
\item
  Conduct a comprehensive assessment of the finance function's AI
  readiness across all countries of operation
\item
  Evaluate data availability, quality, and accessibility
\item
  Assess current technology infrastructure and skills
\item
  Identify high-priority use cases based on potential impact and
  feasibility
\end{itemize}

\paragraph{Data Infrastructure
Preparation}\label{data-infrastructure-preparation}

\begin{itemize}
\tightlist
\item
  Establish data governance framework aligned with the Foundation's
  values and operational needs
\item
  Develop data quality standards and improvement processes
\item
  Create data integration architecture to connect Microsoft Dynamics
  Business Central with other systems
\item
  Implement data security protocols appropriate for financial
  information
\end{itemize}

\paragraph{Pilot Implementations}\label{pilot-implementations}

\begin{itemize}
\tightlist
\item
  Deploy automated document processing for invoices and receipts
\item
  Implement a basic financial analytics dashboard for the CFO and Global
  Finance Lead
\item
  Develop a proof-of-concept for the natural language interface to
  financial data
\end{itemize}

\paragraph{Team Training and Skill
Development}\label{team-training-and-skill-development}

\begin{itemize}
\tightlist
\item
  Conduct AI awareness training for all finance staff
\item
  Provide specialized training for the finance analysts who will be
  working most closely with AI tools
\item
  Establish a community of practice for AI in finance across the
  Foundation
\end{itemize}

\paragraph{Key Milestones and Success
Metrics}\label{key-milestones-and-success-metrics}

\begin{itemize}
\tightlist
\item
  Completion of AI readiness assessment with clear action plan
\item
  Data governance framework established and approved
\item
  At least two pilot implementations successfully deployed
\item
  80\% of finance staff completed basic AI awareness training
\end{itemize}

According to the MITRE AI Maturity Model, ``Organizations in the early
stages of AI adoption should focus on preparing their data
infrastructure and conducting limited pilots to demonstrate value''
(Bloedorn et al., 2023). This approach allows Pharo Foundation to build
momentum while establishing the necessary foundation for more advanced
implementations.

\subsubsection{Phase 2: Expansion (7-18
months)}\label{phase-2-expansion-7-18-months}

The second phase will focus on scaling successful pilots and
implementing more complex AI capabilities:

\paragraph{Scaling Successful Pilots}\label{scaling-successful-pilots}

\begin{itemize}
\tightlist
\item
  Extend document processing automation to all countries of operation
\item
  Expand the financial analytics dashboard to country-level finance
  managers
\item
  Implement full predictive forecasting capabilities for both
  Development and Ventures
\end{itemize}

\paragraph{System Integration}\label{system-integration}

\begin{itemize}
\tightlist
\item
  Integrate AI capabilities with Microsoft Dynamics Business Central
\item
  Connect financial AI tools with the Foundation's cloud infrastructure
  (Microsoft 365)
\item
  Establish connections with HRIS, EMIS, MES, and LMS/HIMS systems to
  enrich financial data with operational context
\end{itemize}

\paragraph{Advanced AI Capability
Implementation}\label{advanced-ai-capability-implementation}

\begin{itemize}
\tightlist
\item
  Deploy comprehensive predictive analytics for financial forecasting
\item
  Implement risk management tools for compliance monitoring
\item
  Develop cross-entity analysis capabilities to support unified
  financial management
\end{itemize}

\paragraph{Team Upskilling and Change
Management}\label{team-upskilling-and-change-management}

\begin{itemize}
\tightlist
\item
  Provide advanced AI skills training for selected finance team members
\item
  Implement change management initiatives to drive adoption
\item
  Establish feedback mechanisms to continuously improve AI tools
\end{itemize}

\paragraph{Key Milestones and Success
Metrics}\label{key-milestones-and-success-metrics-1}

\begin{itemize}
\tightlist
\item
  AI document processing implemented across all countries
\item
  Predictive forecasting achieving accuracy targets (within 10\% of
  actuals)
\item
  At least 50\% reduction in manual data entry for financial
  transactions
\item
  Positive user satisfaction scores (\textgreater80\%) from finance team
  members
\end{itemize}

As noted by Kausel et al.~(2022), ``The expansion phase of AI
implementation requires strong change management to overcome potential
resistance and ensure adoption.'' Pharo Foundation will need to
carefully balance technical implementation with organizational change
management during this phase.

\subsubsection{Phase 3: Optimization (19-36
months)}\label{phase-3-optimization-19-36-months}

The final phase will focus on advanced AI capabilities, continuous
improvement, and knowledge sharing:

\paragraph{AI-Driven Continuous
Improvement}\label{ai-driven-continuous-improvement}

\begin{itemize}
\tightlist
\item
  Implement AI tools that automatically identify process improvement
  opportunities
\item
  Establish automated quality monitoring for financial data and
  processes
\item
  Develop self-optimizing AI models that improve with continued use
\end{itemize}

\paragraph{Advanced Analytics and
Insights}\label{advanced-analytics-and-insights}

\begin{itemize}
\tightlist
\item
  Deploy sophisticated scenario planning capabilities
\item
  Implement impact assessment tools that connect financial decisions to
  mission outcomes
\item
  Develop AI-powered strategic planning support for both arms of the
  Foundation
\end{itemize}

\paragraph{AI Center of Excellence}\label{ai-center-of-excellence}

\begin{itemize}
\tightlist
\item
  Establish a dedicated AI Center of Excellence for finance
\item
  Develop internal AI innovation capabilities
\item
  Create a framework for evaluating and implementing new AI technologies
\end{itemize}

\paragraph{Knowledge Sharing Across
Operations}\label{knowledge-sharing-across-operations}

\begin{itemize}
\tightlist
\item
  Implement AI-powered knowledge management systems
\item
  Establish communities of practice across countries
\item
  Develop training and documentation to support continuous learning
\end{itemize}

\paragraph{Key Milestones and Success
Metrics}\label{key-milestones-and-success-metrics-2}

\begin{itemize}
\tightlist
\item
  AI Center of Excellence established and operational
\item
  At least 70\% of routine financial processes automated
\item
  Financial forecasting accuracy improved to within 5\% of actuals
\item
  Demonstrated link between AI insights and improved program outcomes
\end{itemize}

As Strich et al.~(2021) note, ``The most mature stage of AI
implementation involves embedding AI capabilities throughout the
organization and creating continuous improvement mechanisms.'' By the
end of this phase, AI should be fully integrated into Pharo Foundation's
finance function and delivering measurable value.

\subsubsection{Resource Requirements}\label{resource-requirements}

Successful implementation of this AI strategy will require investments
in technology, talent, and organizational change:

\paragraph{Technology Infrastructure}\label{technology-infrastructure}

\begin{itemize}
\tightlist
\item
  \textbf{Cloud Computing Resources}: Enhanced cloud infrastructure to
  support AI workloads, potentially expanding current Microsoft 365
  capabilities
\item
  \textbf{AI Development Platforms}: Tools and platforms for developing,
  testing, and deploying AI models
\item
  \textbf{Data Storage and Processing}: Expanded data warehouse
  capabilities to support AI analytics
\item
  \textbf{Integration Middleware}: Tools to facilitate data flow between
  systems, especially between Business Central and other operational
  systems
\end{itemize}

\paragraph{Talent and Skills}\label{talent-and-skills}

\begin{itemize}
\tightlist
\item
  \textbf{Data Scientists and AI Specialists}: Either hired directly or
  contracted, to develop and implement AI solutions
\item
  \textbf{Financial Technology Experts}: To bridge the gap between
  finance domain knowledge and AI capabilities
\item
  \textbf{Change Management Professionals}: To support the
  organizational transition to AI-enhanced finance operations
\item
  \textbf{Training Programs}: Comprehensive training to upskill existing
  finance staff
\end{itemize}

\paragraph{Organizational Changes}\label{organizational-changes}

\begin{itemize}
\tightlist
\item
  \textbf{New Roles and Responsibilities}: Creation of new positions
  such as AI Product Owner for Finance and Data Governance Lead
\item
  \textbf{Revised Workflows and Processes}: Redesign of financial
  processes to incorporate AI capabilities
\item
  \textbf{Governance Structures}: Establishment of oversight committees
  for AI implementation and ethics
\item
  \textbf{Cross-Functional Collaboration Mechanisms}: Forums and
  processes to facilitate collaboration between finance, IT, and program
  teams
\end{itemize}

The investment required for this AI strategy implementation should be
viewed in the context of expected returns. According to McKinsey (2024),
``Organizations that successfully implement AI in finance functions can
expect cost reductions of 20-30\% while improving decision quality and
speed.''

\subsubsection{Monitoring and Evaluation
Metrics}\label{monitoring-and-evaluation-metrics}

Drawing from Pharo Foundation's Global Monitoring Framework (GMF), this
AI strategy will be monitored and evaluated using a structured approach
that emphasizes credible, actionable, and responsible data collection
and analysis.

\paragraph{Technical Performance
Metrics}\label{technical-performance-metrics}

\begin{itemize}
\tightlist
\item
  \textbf{AI Model Accuracy}: Measuring the precision and recall of AI
  models against established benchmarks
\item
  \textbf{System Response Times}: Tracking the speed of AI systems in
  responding to user queries and processing data
\item
  \textbf{Data Quality Measurements}: Monitoring data completeness,
  accuracy, and timeliness to ensure AI models have quality inputs
\end{itemize}

\paragraph{Business Impact Metrics}\label{business-impact-metrics}

\begin{itemize}
\tightlist
\item
  \textbf{Process Efficiency Improvements}: Measuring reductions in
  processing time and manual effort
\item
  \textbf{Cost Reduction}: Tracking direct cost savings from AI
  implementation
\item
  \textbf{Revenue Enhancement for Pharo Ventures}: Assessing the impact
  of AI insights on investment returns
\item
  \textbf{Project Effectiveness for Pharo Development}: Measuring
  improved resource allocation efficiency for development projects
\item
  \textbf{User Adoption Rates}: Tracking the percentage of finance staff
  actively using AI tools
\end{itemize}

\paragraph{ROI Calculation Framework}\label{roi-calculation-framework}

\begin{itemize}
\tightlist
\item
  \textbf{Cost-Benefit Analysis}: Comparing implementation costs with
  quantifiable benefits
\item
  \textbf{Value Realization Timeline}: Tracking when benefits are
  realized relative to investments
\item
  \textbf{Success Criteria Definition}: Clear definitions of what
  constitutes success for each AI initiative
\end{itemize}

Drawing from the GMF's approach, the monitoring framework will
incorporate both I-scores (implementation fidelity) and D-scores (design
fidelity) to ensure that AI solutions are not only technically sound but
also meeting user needs. As noted in the GMF, ``While rigor is
important, we recognize that timely recommendations are essential for
enhancing program outcomes and driving future investments'' (Pharo
Foundation GMF, 2024).

\subsection{Challenges and Ethical
Considerations}\label{challenges-and-ethical-considerations}

Implementing AI in Pharo Foundation's finance function presents unique
challenges given the organization's dual structure and multi-country
operations in Africa. These challenges must be addressed proactively to
ensure successful implementation and alignment with the Foundation's
values.

\subsubsection{Implementation
Challenges}\label{implementation-challenges}

\paragraph{Data Quality and Integration
Issues}\label{data-quality-and-integration-issues}

Data challenges are particularly acute in the African context, where
digital infrastructure may be limited:

\begin{itemize}
\item
  \textbf{Addressing Data Silos}: Pharo Development and Pharo Ventures
  currently operate with separate systems and data stores. As noted in
  the finance transformation strategy, creating ``a unified financial
  system that integrates data from both non-profit and for-profit
  operations'' is a key objective but presents significant technical
  challenges.
\item
  \textbf{Standardizing Data Across Countries}: The Foundation operates
  across multiple African countries (Ethiopia, Kenya, Rwanda, and
  Somaliland), each with different financial systems and practices.
  Creating standardized data definitions and formats across these
  contexts will require significant effort.
\item
  \textbf{Implementing Robust Data Governance}: According to Barros Pena
  et al.~(2021), ``Organizations implementing AI in emerging markets
  face significantly greater data governance challenges than those in
  developed economies.'' The Foundation will need to develop data
  governance frameworks that work across its diverse operational
  contexts.
\end{itemize}

To address these challenges, the AI strategy must include strong data
governance components and phased implementation that prioritizes data
quality improvement. As Thurman et al.~(2019) note, ``Without quality
data, even the most sophisticated AI models will deliver limited
value.''

\paragraph{Change Management Hurdles}\label{change-management-hurdles}

Cultural and organizational factors present significant challenges:

\begin{itemize}
\item
  \textbf{Managing Transition from Manual to AI-Assisted Processes}: The
  finance transformation notes highlight that ``legacy workflows are
  hard to let go of'' and users are ``creatures of habit.'' Helping
  finance staff adapt to AI-enhanced workflows will require dedicated
  change management.
\item
  \textbf{Building AI Literacy Across Diverse Teams}: There is a
  ``finance knowledge gap between Kenya and other regions (Ethiopia,
  Rwanda and especially Somaliland)'' noted in the finance
  transformation documentation. This disparity will need to be addressed
  through tailored training programs.
\item
  \textbf{Ensuring User Acceptance and Engagement}: According to
  Edmondson (2019), psychological safety is essential for successful
  technology adoption. The Foundation must create an environment where
  finance staff feel safe experimenting with new AI tools.
\end{itemize}

Successful adoption will require a comprehensive change management
strategy that includes clear communication, targeted training, and
visible leadership support. As Florida (2012) suggests, ``Today's
knowledge workers cannot simply be ordered to adopt new technologies;
they must be motivated to engage.''

\paragraph{Technical Challenges}\label{technical-challenges}

Implementing AI in African contexts presents specific technical
challenges:

\begin{itemize}
\item
  \textbf{Connectivity Issues}: Many of the Foundation's operations are
  in areas with limited internet connectivity, which can impact
  cloud-based AI systems. Solutions must be designed with these
  constraints in mind.
\item
  \textbf{Integration with Existing Systems}: The recent migration to
  Microsoft Dynamics Business Central presents both an opportunity and a
  challenge. As noted in the finance transformation notes, there are
  ``no clear incompany data strategy'' and ``users {[}are{]} unsure of
  new requisition workflows.''
\item
  \textbf{Ensuring Solution Scalability and Reliability}: AI solutions
  must be designed to scale across the Foundation's operations while
  maintaining reliability in varied technical environments.
\end{itemize}

These technical challenges require careful planning and potentially
hybrid approaches that combine cloud and edge computing technologies. As
noted by Jung and Seiter (2021), ``AI implementations in emerging
markets often require adaptation of standard approaches to account for
infrastructure limitations.''

\subsubsection{Ethical Implications}\label{ethical-implications}

Implementing AI in finance also raises important ethical considerations
that align with Pharo Foundation's values of integrity, excellence, and
impact:

\paragraph{Fairness and Bias
Considerations}\label{fairness-and-bias-considerations}

\begin{itemize}
\item
  \textbf{Ensuring Cultural Appropriateness}: AI models must be
  evaluated for cultural appropriateness across the different African
  contexts where Pharo operates. Models trained primarily on Western
  data may not perform equally well in African contexts.
\item
  \textbf{Addressing Potential Biases}: According to Shin (2022b), ``AI
  models can perpetuate existing biases in financial decision-making if
  not carefully designed and monitored.'' The Foundation must implement
  bias detection and mitigation strategies.
\item
  \textbf{Equitable Application}: AI benefits should be distributed
  equitably across Pharo's operations, ensuring that technology does not
  exacerbate existing inequalities between regions or between
  Development and Ventures arms.
\end{itemize}

These considerations require ongoing monitoring and adjustment of AI
systems to ensure they support the Foundation's mission of fostering
self-reliance across Africa.

\paragraph{Accountability Frameworks}\label{accountability-frameworks}

\begin{itemize}
\item
  \textbf{Clear Responsibility Structures}: The Foundation must
  establish clear accountability for AI decisions, particularly those
  related to financial resource allocation or risk assessment.
\item
  \textbf{Oversight Mechanisms}: Implementing governance structures that
  provide appropriate oversight of AI systems while maintaining
  operational efficiency.
\item
  \textbf{Audit Procedures}: Developing procedures for regular auditing
  of AI systems to ensure they operate as intended and align with
  organizational values.
\end{itemize}

Shin et al.~(2022) emphasize that ``algorithmic transparency and
accountability are especially important in contexts where technology
adoption may outpace regulatory frameworks.'' This is particularly
relevant in the African countries where Pharo operates.

\paragraph{Societal Impact}\label{societal-impact}

\begin{itemize}
\item
  \textbf{Effect on Workforce Roles and Skills}: AI implementation will
  change the nature of finance roles at Pharo Foundation. According to
  McKinsey (2023), ``AI in finance functions typically shifts roles from
  transaction processing to analysis and decision support.''
\item
  \textbf{Alignment with Mission of Self-Reliance}: All AI
  implementations should ultimately support Pharo's mission of fostering
  self-reliance in Africa, ensuring technology builds rather than
  replaces local capacity.
\item
  \textbf{Contribution to Sustainable Development Goals}: AI
  applications should be evaluated for their contribution to sustainable
  development in the regions where Pharo operates.
\end{itemize}

As Pharo Foundation aims to create ``a vibrant, productive, and
self-reliant Africa,'' AI implementation must be thoughtfully designed
to enhance rather than diminish human capabilities. As noted by Longoni
et al.~(2019), ``Technology implementations are most successful when
they augment rather than replace human judgment and expertise.''

\subsection{Conclusion and
Recommendations}\label{conclusion-and-recommendations}

Pharo Foundation stands at a pivotal moment in its finance
transformation journey, with artificial intelligence offering powerful
capabilities to enhance efficiency, improve decision-making, and
strengthen governance. By strategically implementing AI across its
finance functions, the Foundation can better support its mission of
fostering a ``vibrant, productive, and self-reliant Africa.''

\subsubsection{Key Strategic
Recommendations}\label{key-strategic-recommendations}

\begin{enumerate}
\def\labelenumi{\arabic{enumi}.}
\item
  \textbf{Prioritize Data Foundation}: Before implementing advanced AI
  capabilities, ensure the Foundation has strong data governance,
  quality processes, and integration architecture. As noted by Shin
  (2022a), ``The quality of AI outputs is directly dependent on the
  quality of data inputs.''
\item
  \textbf{Take a Phased Approach}: Follow the three-phase implementation
  roadmap, starting with high-value, low-complexity use cases like
  document processing automation before progressing to more
  sophisticated applications. This approach aligns with best practices
  identified by Bloedorn et al.~(2023) in the MITRE AI Maturity Model.
\item
  \textbf{Invest in People}: Complement technology investments with
  comprehensive training and change management to build AI literacy
  across the finance function. As Deci and Flaste (1995) emphasize,
  intrinsic motivation is essential for successful technology adoption.
\item
  \textbf{Ensure Cross-Entity Integration}: Design AI solutions that
  bridge Pharo Development and Pharo Ventures, supporting the
  Foundation's goal of a unified financial system that enables seamless
  data flow and coordination across entities, business units, and
  countries.
\item
  \textbf{Embed Ethical Considerations}: Integrate fairness,
  accountability, and alignment with Pharo's mission into all AI
  implementations, ensuring technology enhances rather than undermines
  the Foundation's values and goals.
\end{enumerate}

\subsubsection{Critical Success Factors}\label{critical-success-factors}

For successful AI implementation, the Foundation should focus on these
critical factors:

\begin{itemize}
\tightlist
\item
  \textbf{Executive Sponsorship}: Active support from the CFO and Global
  Finance Lead to drive change and remove barriers
\item
  \textbf{Cross-Functional Collaboration}: Close partnership between
  finance, IT, and program teams
\item
  \textbf{User-Centered Design}: Development of AI solutions with
  continuous input from finance staff
\item
  \textbf{Robust Governance}: Clear structures for decision-making, risk
  management, and ethical oversight
\item
  \textbf{Continuous Learning}: Mechanisms to capture lessons learned
  and improve implementation approaches
\end{itemize}

\subsubsection{Long-term Vision}\label{long-term-vision}

By successfully implementing this AI strategy, Pharo Foundation can
transform its finance function from a transaction-processing and
reporting center to a strategic partner that provides data-driven
insights to drive mission impact. AI-enhanced finance capabilities will
support more efficient resource allocation, stronger risk management,
and ultimately greater impact across Pharo's education, water, and
productivity missions.

As the Foundation works to foster self-reliance in Africa, AI can serve
as a powerful enabler, helping to optimize the use of resources, improve
transparency, and strengthen decision-making. With thoughtful
implementation that balances technological innovation with
organizational and ethical considerations, Pharo Foundation can leverage
AI to amplify its contribution to sustainable development across Africa.

\section*{References}\label{references}
\addcontentsline{toc}{section}{References}

\phantomsection\label{refs}
\begin{CSLReferences}{1}{0}
\bibitem[\citeproctext]{ref-barrospena2021circumspect}
Barros Pena, B., Clarke, R. E., Holmquist, L. E., \& Vines, J. (2021).
Circumspect users: Older adults as critical adopters and resistors of
technology. \emph{Proceedings of the CHI Conference on Human Factors in
Computing Systems}.

\bibitem[\citeproctext]{ref-bloedorn2023mitre}
Bloedorn, E. E., Kotras, D. M., Schwartz, P. J., Chaney, C., Chaney, C.,
\& Patsis, J. (2023). \emph{The MITRE AI maturity model and
organizational assessment tool guide: A path to successful AI adoption}.
MITRE.

\bibitem[\citeproctext]{ref-brynjolfsson2017what}
Brynjolfsson, E., \& Mitchell, T. (2017). What can machine learning do?
Workforce implications. \emph{Science}, \emph{358}(6370), 1530--1534.

\bibitem[\citeproctext]{ref-castelo2019task}
Castelo, N., Bos, M. W., \& Lehmann, D. R. (2019). Task-dependent
algorithm aversion. \emph{Journal of Marketing Research}, \emph{56}(5),
809--825.

\bibitem[\citeproctext]{ref-collins2021artificial}
Collins, C., Dennehy, D., Conboy, K., \& Mikalef, P. (2021). Artificial
intelligence in information systems research: A systematic literature
review and research agenda. \emph{International Journal of Information
Management}, \emph{60}, 102383.

\bibitem[\citeproctext]{ref-davenport2018artificial}
Davenport, T. H., \& Ronanki, R. (2018). Artificial intelligence for the
real world. \emph{Harvard Business Review}, \emph{96}(1), 108--116.

\bibitem[\citeproctext]{ref-deci1995why}
Deci, E. L., \& Flaste, R. (1995). \emph{Why we do what we do}. Penguin
Books.

\bibitem[\citeproctext]{ref-edmondson2019fearless}
Edmondson, A. (2019). \emph{The fearless organization: Creating
psychological safety in the workplace for learning, innovation, and
growth}. John Wiley \& Sons.

\bibitem[\citeproctext]{ref-florida2012creative}
Florida, R. (2012). \emph{The rise of the creative class, revisited:
Revised and expanded}. Basic Books.

\bibitem[\citeproctext]{ref-iansiti2020competing}
Iansiti, M., \& Lakhani, K. R. (2020). Competing in the age of AI.
\emph{Harvard Business Review}, \emph{98}(1), 60--67.

\bibitem[\citeproctext]{ref-jung2021towards}
Jung, M., \& Seiter, M. (2021). Towards a better understanding on
mitigating algorithm aversion in forecasting: An experimental study.
\emph{Journal of Management Control}, \emph{32}, 495--516.

\bibitem[\citeproctext]{ref-kausel2022longitudinal}
Kausel, E. E., Reyes, T., \& Chacon, A. (2022). A longitudinal approach
for understanding algorithm use. \emph{Journal of Behavioral Decision
Making}, \emph{35}(4), e2275.

\bibitem[\citeproctext]{ref-lee2004trust}
Lee, J. D., \& See, K. A. (2004). Trust in automation: Designing for
appropriate reliance. \emph{Human Factors}, \emph{46}(1), 50--80.

\bibitem[\citeproctext]{ref-longoni2019resistance}
Longoni, C., Bonezzi, A., \& Morewedge, C. K. (2019). Resistance to
medical artificial intelligence. \emph{Journal of Consumer Research},
\emph{46}(4), 629--650.

\bibitem[\citeproctext]{ref-maedche2019ai}
Maedche, A., Legner, C., Benlian, A., Berger, B., Gimpel, H., Hess, T.,
Hinz, O., Morana, S., \& Söllner, M. (2019). AI-based digital
assistants. \emph{Business \& Information Systems Engineering},
\emph{61}(4), 535--544.

\bibitem[\citeproctext]{ref-mckinsey2023state}
McKinsey \& Company. (2023). \emph{The state of AI in early 2024: Gen AI
adoption spikes and starts to generate value}.

\bibitem[\citeproctext]{ref-mckinsey2024data}
McKinsey Technology. (2024). \emph{A data leader's operating guide to
scaling gen AI}.

\bibitem[\citeproctext]{ref-mikalef2021artificial}
Mikalef, P., \& Gupta, M. (2021). Artificial intelligence capability:
Conceptualization, measurement calibration, and empirical study on its
impact on organizational creativity and firm performance.
\emph{Information \& Management}, \emph{58}(3), 103434.

\bibitem[\citeproctext]{ref-pharo2023about}
Pharo Foundation. (2023). \emph{About pharo foundation}.
\url{https://pharofoundation.org/our-story}.

\bibitem[\citeproctext]{ref-pharo2024gmf}
Pharo Foundation. (2024). \emph{Global monitoring framework: An
impact-driven approach to design, monitoring, and evaluation}.

\bibitem[\citeproctext]{ref-pharo2024finance}
Pharo Foundation Finance Transformation Strategy. (2024). \emph{Internal
strategic document}.

\bibitem[\citeproctext]{ref-shin2022a}
Shin, D. (2022a). Expanding the role of trust in the experience of
algorithmic journalism: User sensemaking of algorithmic heuristics in
korean users. \emph{Journalism Practice}, \emph{16}(6), 1168--1191.

\bibitem[\citeproctext]{ref-shin2022b}
Shin, D. (2022b). How do people judge the credibility of algorithmic
sources? \emph{AI \& Society}, \emph{37}(1), 81--96.

\bibitem[\citeproctext]{ref-shin2022understanding}
Shin, D., Lim, J. S., Ahmad, N., \& Ibahrine, M. (2022). Understanding
user sensemaking in fairness and transparency in algorithms: Algorithmic
sensemaking in over-the-top platform. \emph{AI \& Society}, 1--14.

\bibitem[\citeproctext]{ref-strich2021what}
Strich, F., Mayer, A. S., \& Fiedler, M. (2021). What do i do in a world
of artificial intelligence? Investigating the impact of substitutive
decision-making AI systems on employees' professional role identity.
\emph{Journal of the Association for Information Systems}, \emph{22}(2),
304--324.

\bibitem[\citeproctext]{ref-thurman2019my}
Thurman, N., Moeller, J., Helberger, N., \& Trilling, D. (2019). My
friends, editors, algorithms, and i. \emph{Digital Journalism},
\emph{7}(4), 447--469.

\end{CSLReferences}




\end{document}
