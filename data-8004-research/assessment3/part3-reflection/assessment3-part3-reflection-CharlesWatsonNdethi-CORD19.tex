% Options for packages loaded elsewhere
\PassOptionsToPackage{unicode}{hyperref}
\PassOptionsToPackage{hyphens}{url}
\PassOptionsToPackage{dvipsnames,svgnames,x11names}{xcolor}
%
\documentclass[
]{article}

\usepackage{amsmath,amssymb}
\usepackage{iftex}
\ifPDFTeX
  \usepackage[T1]{fontenc}
  \usepackage[utf8]{inputenc}
  \usepackage{textcomp} % provide euro and other symbols
\else % if luatex or xetex
  \usepackage{unicode-math}
  \defaultfontfeatures{Scale=MatchLowercase}
  \defaultfontfeatures[\rmfamily]{Ligatures=TeX,Scale=1}
\fi
\usepackage{lmodern}
\ifPDFTeX\else  
    % xetex/luatex font selection
\fi
% Use upquote if available, for straight quotes in verbatim environments
\IfFileExists{upquote.sty}{\usepackage{upquote}}{}
\IfFileExists{microtype.sty}{% use microtype if available
  \usepackage[]{microtype}
  \UseMicrotypeSet[protrusion]{basicmath} % disable protrusion for tt fonts
}{}
\makeatletter
\@ifundefined{KOMAClassName}{% if non-KOMA class
  \IfFileExists{parskip.sty}{%
    \usepackage{parskip}
  }{% else
    \setlength{\parindent}{0pt}
    \setlength{\parskip}{6pt plus 2pt minus 1pt}}
}{% if KOMA class
  \KOMAoptions{parskip=half}}
\makeatother
\usepackage{xcolor}
\usepackage[margin=1in]{geometry}
\setlength{\emergencystretch}{3em} % prevent overfull lines
\setcounter{secnumdepth}{5}
% Make \paragraph and \subparagraph free-standing
\makeatletter
\ifx\paragraph\undefined\else
  \let\oldparagraph\paragraph
  \renewcommand{\paragraph}{
    \@ifstar
      \xxxParagraphStar
      \xxxParagraphNoStar
  }
  \newcommand{\xxxParagraphStar}[1]{\oldparagraph*{#1}\mbox{}}
  \newcommand{\xxxParagraphNoStar}[1]{\oldparagraph{#1}\mbox{}}
\fi
\ifx\subparagraph\undefined\else
  \let\oldsubparagraph\subparagraph
  \renewcommand{\subparagraph}{
    \@ifstar
      \xxxSubParagraphStar
      \xxxSubParagraphNoStar
  }
  \newcommand{\xxxSubParagraphStar}[1]{\oldsubparagraph*{#1}\mbox{}}
  \newcommand{\xxxSubParagraphNoStar}[1]{\oldsubparagraph{#1}\mbox{}}
\fi
\makeatother


\providecommand{\tightlist}{%
  \setlength{\itemsep}{0pt}\setlength{\parskip}{0pt}}\usepackage{longtable,booktabs,array}
\usepackage{calc} % for calculating minipage widths
% Correct order of tables after \paragraph or \subparagraph
\usepackage{etoolbox}
\makeatletter
\patchcmd\longtable{\par}{\if@noskipsec\mbox{}\fi\par}{}{}
\makeatother
% Allow footnotes in longtable head/foot
\IfFileExists{footnotehyper.sty}{\usepackage{footnotehyper}}{\usepackage{footnote}}
\makesavenoteenv{longtable}
\usepackage{graphicx}
\makeatletter
\newsavebox\pandoc@box
\newcommand*\pandocbounded[1]{% scales image to fit in text height/width
  \sbox\pandoc@box{#1}%
  \Gscale@div\@tempa{\textheight}{\dimexpr\ht\pandoc@box+\dp\pandoc@box\relax}%
  \Gscale@div\@tempb{\linewidth}{\wd\pandoc@box}%
  \ifdim\@tempb\p@<\@tempa\p@\let\@tempa\@tempb\fi% select the smaller of both
  \ifdim\@tempa\p@<\p@\scalebox{\@tempa}{\usebox\pandoc@box}%
  \else\usebox{\pandoc@box}%
  \fi%
}
% Set default figure placement to htbp
\def\fps@figure{htbp}
\makeatother

\usepackage{placeins}
\usepackage{needspace}
\usepackage{afterpage}

% Prevent widows and orphans (single lines at the beginning or end of a page)
\widowpenalty=10000
\clubpenalty=10000

% Keep section headings with the following paragraph
\usepackage[nobottomtitles*]{titlesec}
\titlespacing{\section}{0pt}{*4}{*2}

% Keep figures where they should be
\usepackage{float}

% Add minimum space before new section to avoid page breaks right after headings
\newcommand{\needsection}[1]{\needspace{2\baselineskip}#1}
\makeatletter
\@ifpackageloaded{caption}{}{\usepackage{caption}}
\AtBeginDocument{%
\ifdefined\contentsname
  \renewcommand*\contentsname{Table of contents}
\else
  \newcommand\contentsname{Table of contents}
\fi
\ifdefined\listfigurename
  \renewcommand*\listfigurename{List of Figures}
\else
  \newcommand\listfigurename{List of Figures}
\fi
\ifdefined\listtablename
  \renewcommand*\listtablename{List of Tables}
\else
  \newcommand\listtablename{List of Tables}
\fi
\ifdefined\figurename
  \renewcommand*\figurename{Figure}
\else
  \newcommand\figurename{Figure}
\fi
\ifdefined\tablename
  \renewcommand*\tablename{Table}
\else
  \newcommand\tablename{Table}
\fi
}
\@ifpackageloaded{float}{}{\usepackage{float}}
\floatstyle{ruled}
\@ifundefined{c@chapter}{\newfloat{codelisting}{h}{lop}}{\newfloat{codelisting}{h}{lop}[chapter]}
\floatname{codelisting}{Listing}
\newcommand*\listoflistings{\listof{codelisting}{List of Listings}}
\makeatother
\makeatletter
\makeatother
\makeatletter
\@ifpackageloaded{caption}{}{\usepackage{caption}}
\@ifpackageloaded{subcaption}{}{\usepackage{subcaption}}
\makeatother

\usepackage{bookmark}

\IfFileExists{xurl.sty}{\usepackage{xurl}}{} % add URL line breaks if available
\urlstyle{same} % disable monospaced font for URLs
\hypersetup{
  pdftitle={Reflection on Learning: The Journey Through Research Methods and Tools},
  pdfauthor={Charles Watson Ndethi Kibaki},
  colorlinks=true,
  linkcolor={blue},
  filecolor={Maroon},
  citecolor={Blue},
  urlcolor={Blue},
  pdfcreator={LaTeX via pandoc}}


\title{Reflection on Learning: The Journey Through Research Methods and
Tools}
\author{Charles Watson Ndethi Kibaki}
\date{2025-04-10}

\begin{document}
\maketitle
\begin{abstract}
This reflection documents my transformative journey through the Research
Methods and Tools course, chronicling the evolution from viewing
research as an intimidating, mathematically-focused endeavor to
understanding it as a structured process of inquiry accessible to all
dedicated learners. I discuss my development of critical research skills
including literature review proficiency, methodological literacy, and
ethical awareness, particularly as applied to my interest in NLP for
low-resource African languages. The document explores challenges faced
in navigating methodological complexity and time management, and
outlines how these lessons will inform my future academic work. Through
this metacognitive process, I've gained confidence in my ability to
conduct meaningful research that aligns with my personal interests and
has real-world impact for underrepresented communities.
\end{abstract}


\section{Introduction}\label{introduction}

When I began the Research Methods and Tools course, I approached it with
a mixture of anticipation and apprehension. My previous encounters with
research methodologies during my undergraduate studies, particularly
struggling through Calculus classes, had left me with the impression
that research was a domain reserved for mathematical prodigies and
statistical virtuosos. This perception was further reinforced when I was
advised against pursuing a data science graduate course due to concerns
about my quantitative abilities. Entering this course, I carried these
doubts with me, unsure if I possessed the necessary aptitude to engage
meaningfully with formal research methods.

What I discovered instead was a transformative journey that
fundamentally changed how I approach knowledge acquisition, analysis,
and contribution to academic discourse. I learned that while statistics
and quantitative methods are important components of research, they are
merely tools within a broader framework of inquiry. The course revealed
the multifaceted nature of research methodology---from choosing research
questions to selecting appropriate instruments and applying existing
analytical frameworks. In today's era of artificial intelligence, many
technical aspects that once seemed intimidating can be abstracted away,
allowing researchers to focus on the conceptual and design elements that
drive meaningful inquiry.

This reflection documents this journey, from theoretical understanding
to practical application, highlighting key learning moments, challenges
overcome, and the evolution of my perspective as a researcher. Through
this process of metacognition---thinking about my own thinking and
learning---I hope to consolidate my growth and establish a foundation
for my continuing development as a researcher in the field of
responsible AI and low-resource natural language processing (NLP).

\section{Evolution of Research
Understanding}\label{evolution-of-research-understanding}

\subsection{From Implementation to
Investigation}\label{from-implementation-to-investigation}

My background in technology implementation had conditioned me to
approach problems with a solution-oriented mindset: identify a technical
issue, research existing solutions, implement and test. This approach,
while effective for development work, lacks the rigor and systematic
inquiry that characterizes academic research. Through this course, I've
transitioned from viewing research as merely ``finding information'' to
understanding it as a structured process of inquiry, evidence-gathering,
and critical analysis.

A pivotal moment in this evolution came during Week 2's literature
review sessions, when I realized that effective research isn't just
about finding answers but about asking the right questions. The
structured approach to formulating research questions using criteria
like specificity, measurability, and relevance transformed how I
conceptualized my research interests. Rather than thinking broadly about
``improving NLP for African languages,'' I learned to formulate
specific, answerable questions like ``What patterns of linguistic
diversity exist within the CORD-19 dataset, and what representation gaps
can be identified for low-resource languages?''

This shift from implementation-focused thinking to investigation-focused
thinking represents perhaps the most fundamental change in my approach
to academic work. I now understand research not as a means to an
immediate practical end, but as a rigorous process of building and
contributing to knowledge that ultimately serves broader aims.

\subsection{Developing Methodological
Literacy}\label{developing-methodological-literacy}

Prior to this course, my understanding of research methodologies was
limited and largely intuitive rather than formal. Terms like
``qualitative analysis,'' ``experimental design,'' and ``correlation
research'' were familiar but not clearly differentiated in my mind. The
course provided a comprehensive framework for understanding different
methodological approaches and their appropriate applications.

Week 4's exploration of qualitative versus quantitative research was
particularly enlightening. I had previously assumed that quantitative
approaches were inherently more ``scientific'' or rigorous, reflecting a
bias common in technical fields. Learning about the strengths of
qualitative methods for exploring complex phenomena and generating rich
insights helped me appreciate the complementary nature of different
methodological traditions.

This methodological literacy proved invaluable when designing my
research proposal on linguistic diversity in the CORD-19 dataset. I was
able to make informed decisions about mixed-methods approaches,
combining quantitative analysis of language distribution with
qualitative assessment of content characteristics. Rather than
defaulting to purely computational methods, I developed a more nuanced
approach that could address both the ``what'' and ``why'' questions
about language representation gaps.

\section{Development of Critical Research
Skills}\label{development-of-critical-research-skills}

\subsection{Literature Review
Proficiency}\label{literature-review-proficiency}

The development of literature review skills represented one of my most
significant areas of growth. Prior to this course, my approach to
literature was haphazard and often focused on confirming existing ideas
rather than critically engaging with diverse perspectives. The CRAAP
framework (Currency, Relevance, Authority, Accuracy, Purpose) introduced
in Week 3 transformed how I evaluate research sources.

Applying this framework to my research on NLP for low-resource languages
led me to discover critical perspectives I had previously overlooked.
For example, examining the authority and purpose of sources led me to
Joshi et al.'s (2020) taxonomy of language resource levels and Nekoto et
al.'s (2020) participatory research approach for African languages.
These sources fundamentally shaped my understanding of the challenges
facing low-resource languages and potential methodological approaches to
address them.

Perhaps most importantly, I learned to approach literature not as a
collection of facts to be compiled, but as an ongoing scholarly
conversation to which I could contribute. This shift from passive
consumption to active engagement with literature enabled me to position
my own research questions within existing knowledge gaps, making my work
more relevant and impactful.

\subsection{Data Collection and Analysis
Techniques}\label{data-collection-and-analysis-techniques}

The course provided comprehensive exposure to both qualitative and
quantitative data collection methods. Week 6's focus on questionnaire
design principles proved unexpectedly valuable for my CORD-19 analysis
project. Although I was working with existing data rather than
collecting new information, the principles of clarity, specificity, and
unbiased question formulation directly informed how I structured my
content analysis categories.

My data analysis skills developed substantially through both theoretical
understanding and practical application. The statistical concepts
covered in Week 7 helped me interpret patterns in the CORD-19 language
distribution with greater sophistication. Rather than simply reporting
frequency counts, I learned to apply correlation analyses and
representation indices that provided more meaningful insights into
linguistic diversity patterns.

A particular breakthrough came when applying text complexity analysis
across different languages in the dataset. Initially, I struggled with
comparing readability metrics across languages that use different
scripts and structural patterns. The course's emphasis on methodology
adaptation for different research contexts encouraged me to develop
language-agnostic complexity metrics based on sentence structure and
terminology density, resulting in more valid cross-linguistic
comparisons.

\subsection{Ethical Research
Practices}\label{ethical-research-practices}

Perhaps the most profound development in my research approach has been a
heightened awareness of ethical considerations. Week 10's focus on
plagiarism and intellectual property expanded my understanding of
ethical research beyond simply avoiding direct copying to encompass
proper attribution, respect for others' ideas, and transparent reporting
of findings.

More broadly, the course heightened my awareness of the ethical
implications of research choices throughout the entire research process.
When analyzing the CORD-19 dataset, this awareness led me to consider
not just technical questions about language identification, but broader
questions about what representation gaps might mean for information
equity and global health outcomes.

This ethical sensitivity is particularly important for my interest in
African language NLP, where issues of data sovereignty, community
involvement, and potential unintended consequences of technology
deployment are critical considerations. I now approach these questions
not as tangential concerns but as central to responsible research
practice.

\section{Methodological Challenges and
Solutions}\label{methodological-challenges-and-solutions}

\subsection{Navigating Methodological
Complexity}\label{navigating-methodological-complexity}

One of the most significant challenges I encountered was navigating the
complexity of methodological choices. The range of potential approaches,
each with its strengths and limitations, sometimes felt overwhelming.
This was particularly evident when designing my research proposal on
linguistic diversity in the CORD-19 dataset. I initially struggled to
determine whether a primarily quantitative, qualitative, or
mixed-methods approach would be most appropriate.

I overcame this challenge through iterative refinement of my research
questions and careful consideration of which methodologies would best
address each aspect of my inquiry. The course's emphasis on aligning
methodology with research questions rather than forcing questions to fit
preferred methods was invaluable in this process. By focusing first on
what I wanted to know and then selecting appropriate methods, I
developed a more coherent and effective research design.

This experience taught me that methodological complexity is not a
barrier to be avoided but a resource to be leveraged. The diversity of
research approaches provides flexibility to address different types of
questions and triangulate findings through multiple methods. Rather than
seeing methodological decisions as binary choices, I now approach them
as strategic selections from a continuum of options based on specific
research needs.

\subsection{Balancing Depth and
Breadth}\label{balancing-depth-and-breadth}

Another significant challenge was balancing depth and breadth in my
research. When analyzing the CORD-19 dataset, I was initially tempted to
explore every possible aspect of linguistic diversity, from language
distribution to content characteristics, text complexity, and named
entity patterns. While comprehensive in scope, this approach risked
superficial treatment of each aspect.

The solution came through Week 5's discussion of experimental design
principles, particularly the importance of clearly defined variables and
controlled scope. I realized that effective research often requires
strategic focusing rather than exhaustive coverage. By prioritizing key
aspects of linguistic diversity that most directly addressed
representation gaps for low-resource languages, I was able to conduct
more meaningful analysis within the available time and resources.

This lesson in strategic focus has important implications for my
capstone project on African language NLP. Rather than attempting to
address all aspects of low-resource language processing, I now
understand the value of identifying specific, well-defined research
questions that can be investigated with appropriate depth. This approach
is likely to yield more substantial contributions than a broader but
shallower investigation.

\subsection{Computational Challenges in NLP
Research}\label{computational-challenges-in-nlp-research}

A significant challenge I encountered was confronting the computational
demands of natural language processing tasks. My work with the CORD-19
dataset revealed how resource-intensive NLP research can be,
particularly when analyzing documents across multiple languages. Even
basic tasks like multilingual text preprocessing and language
identification required substantial computation time and occasionally
exceeded available memory resources.

These limitations became particularly evident when implementing
cross-lingual analysis techniques. What appeared conceptually
straightforward in research papers often proved computationally
intractable without specialized hardware. I found myself constantly
balancing methodological rigor against practical constraints, making
strategic decisions about sampling approaches to work within available
resources.

This experience has important implications for my future research in
low-resource African languages, where computational efficiency becomes
even more critical. It also highlighted broader equity issues in NLP
research, where access to computational resources often determines which
languages receive attention. Moving forward, I plan to explore more
efficient methodologies while seeking collaborative arrangements that
can provide access to necessary computing infrastructure for work with
underrepresented languages.

\subsection{The Role of AI in
Research}\label{the-role-of-ai-in-research}

Throughout this course, I've developed a nuanced understanding of
artificial intelligence's role in the research process. While AI tools
can significantly assist with data analysis, literature review, and even
writing tasks, they cannot replace human judgment in defining research
objectives and interpreting results within meaningful contexts.

This insight became particularly clear during my CORD-19 analysis
project, where I used AI-assisted tools for language identification and
content classification. While these tools enhanced efficiency, the
critical decisions about what questions to ask, which methodologies to
apply, and how to interpret the findings in relation to my research
questions remained fundamentally human tasks. AI functioned as an
amplifier of my research capabilities rather than a substitute for my
cognitive engagement with the material.

Professor Mouzonni's guidance on AI use was especially valuable---he
emphasized using AI as a thought partner after initial human ideation
rather than as a primary idea generator. This advice proved remarkably
insightful as I discovered through experience that current AI systems,
built on transformer architectures, generally struggle to produce
genuine novelty. As Marcus and Davis (2020) argue, these systems excel
at pattern recognition and recombination of existing knowledge but lack
the conceptual understanding and creative leaps that characterize human
innovation. I found that AI functions better as a ``regurgitation
machine'' that can elaborate on and refine human-generated ideas rather
than as a source of truly original concepts. This reinforced my
understanding that authentic creativity and innovative research
questions remain primarily the domain of human researchers, while AI can
serve as a powerful amplifier and refinement tool for these
human-originated ideas.

I've learned that effective AI integration in research requires clearly
defined objectives and careful oversight. The quality of AI-generated
outputs depends heavily on the quality of human inputs and direction.
This realization has helped me view AI as a powerful collaborator rather
than either a threat to research integrity or a magical solution to all
research challenges.

\section{Research Planning and Time
Management}\label{research-planning-and-time-management}

\subsection{The Importance of Structured
Planning}\label{the-importance-of-structured-planning}

Effective research planning emerged as a crucial skill throughout this
course. Prior to these studies, my approach to project planning was
often reactive and ad hoc, responding to immediate needs rather than
strategically organizing the entire research process. The course's
emphasis on structured research design and timeline development in Week
5 transformed my approach to planning.

For my CORD-19 analysis project, I implemented a phased research plan
with clearly defined milestones for literature review, data acquisition,
preliminary analysis, in-depth investigation, and synthesis of findings.
This structured approach allowed me to track progress and adjust
strategies when certain phases, particularly data preprocessing,
required more time than initially anticipated.

I've learned that effective research planning is not about creating
rigid schedules but about developing flexible frameworks that
accommodate discovery and refinement while maintaining overall
direction. This balance between structure and adaptability will be
crucial for my future research endeavors, including my capstone project.

\subsection{Time Management
Challenges}\label{time-management-challenges}

Time management emerged as a practical challenge throughout my research
projects. The iterative nature of research, with its cycles of
literature review, methodology refinement, data analysis, and
interpretation, requires careful planning to ensure completion within
defined timeframes.

I initially underestimated the time required for certain research
phases, particularly data preprocessing and preliminary analysis. For
the CORD-19 project, language identification and text cleaning took
nearly twice as long as anticipated, creating pressure on subsequent
analysis phases. This experience taught me the importance of building
buffer time into research schedules and periodically reassessing
timelines based on emerging insights.

The course's emphasis on research planning provided valuable strategies
for more effective time management. I've learned to view research
planning not as a fixed linear process but as an adaptive framework that
accommodates discovery and refinement. For future projects, including my
capstone, I plan to implement more explicit milestone tracking and
regular progress assessment to ensure timely completion without
sacrificing research quality.

I will take special regard for Professor Mouzonni's caution about time
management and put into practice his particularly insightful advice that
follows a logarithmic curve---investing more rigor and grit upfront in
the research process, then easing into a more relaxed pace toward thesis
defense season. This approach acknowledges that the early phases of
research, including literature review, methodology design, and data
collection, require intensive effort and meticulous attention to
establish a solid foundation. With this groundwork properly laid, the
later stages of analysis, writing, and revision can proceed more
smoothly and confidently. This logarithmic distribution of effort will
likely lead to higher quality outcomes while reducing stress during the
final stages of project completion.

\section{Application to Future
Research}\label{application-to-future-research}

\subsection{Direct Methodological
Applications}\label{direct-methodological-applications}

The methodologies and techniques learned throughout this course have
direct applications to my capstone project on NLP for low-resource
African languages. The literature review framework will enable me to
systematically map existing work in this field, identifying key gaps and
opportunities for contribution. The mixed-methods approach demonstrated
in my CORD-19 analysis provides a template for combining computational
techniques with qualitative assessment of language representation and
content characteristics.

Particularly valuable will be the experimental design principles covered
in Week 5, which will help structure evaluations of transfer learning
approaches for cross-lingual information retrieval between high-resource
and low-resource languages. The emphasis on clearly defined variables,
appropriate controls, and replicable procedures will strengthen the
validity of these evaluations.

The data collection and analysis techniques from Weeks 6-8 will inform
both corpus development for African languages and subsequent analysis of
language patterns. The understanding of sampling strategies will be
particularly valuable for creating representative datasets within the
constraints of limited available content for low-resource languages.

\subsection{Research with Purpose and
Context}\label{research-with-purpose-and-context}

Perhaps the most significant insight I'll carry forward is the
importance of researching topics that deeply resonate with my personal
interests and cultural context. My passion for improving NLP resources
for underrepresented African languages stems from my own experiences and
cultural background. This connection creates an intrinsic motivation
that makes the research process more meaningful and sustainable, even
when facing challenges.

As Professor Mouzonni emphasized throughout the course, research is not
merely an academic exercise but a foundational approach to creating
impactful work. By focusing on questions that have real-world
significance for communities I care about, I can ensure that my research
contributes meaningfully to both academic knowledge and practical
applications. This purpose-driven approach will guide my selection of
research topics and methodologies in future academic and professional
endeavors.

\subsection{Continuing Education and
Collaboration}\label{continuing-education-and-collaboration}

I recognize that my development as a researcher is an ongoing journey
rather than a destination reached at the conclusion of this course. I
plan to revisit the course materials and practice exercises as I embark
on my capstone project, using them as reference points for
methodological decisions and research planning.

Additionally, I value the importance of maintaining open communication
channels with mentors like Professor Mouzonni, whose patient and
approachable teaching style helped demystify research methodology and
make it accessible. This experience has shown me that research is not
reserved for an elite few with exceptional mathematical abilities, but
is a disciplined approach to inquiry that can be learned and applied by
anyone with dedication and proper guidance.

\section{Recommendations for Course
Improvement}\label{recommendations-for-course-improvement}

While the course was generally well-structured and informative, I
believe it could be enhanced in two key areas. First, more personalized
feedback on assignments would significantly benefit students'
development as researchers. I understand the challenge this presents
given the number of students and the depth of analysis required, but
even brief, targeted comments on methodology choices or analytical
approaches would provide valuable guidance. Perhaps a rotating system
where each assignment receives more detailed feedback, or leveraging AI
tools to provide initial assessment followed by human refinement, could
make this more manageable. Second, incorporating mandatory presentation
sessions where students must explain their research designs and
preliminary findings would strengthen their ability to communicate
research effectively. As preparation for thesis defense and professional
research communication, these opportunities to present work---even in
brief 5-10 minute sessions---would build confidence and clarity in
articulating complex methodological choices. These sessions could be
structured as peer-review exercises to maximize learning while managing
time constraints, with students providing constructive feedback to one
another under faculty guidance. These enhancements would further develop
two critical research skills that complement the excellent
methodological foundation already provided: receiving and incorporating
targeted feedback, and effectively communicating complex research
designs to diverse audiences

\section{Conclusion}\label{conclusion}

My journey through the Research Methods and Tools course has transformed
my understanding of research from an intimidating, statistically-focused
endeavor to a multifaceted process of structured inquiry that can be
approached from various methodological perspectives. I've developed
critical research skills that will serve as a foundation for my academic
and professional work, particularly in my areas of interest related to
NLP for low-resource African languages.

The challenges I've faced---from navigating methodological complexity to
managing research timelines---have provided valuable learning
experiences that have strengthened my capabilities as a researcher. By
finding effective solutions to these challenges, I've gained confidence
in my ability to design and execute meaningful research projects.

Looking ahead, I will apply these insights to my capstone project and
beyond, approaching research not merely as an academic requirement but
as a fundamental tool for creating knowledge with purpose and impact. I
am particularly committed to continuing research that supports
linguistic diversity and representation in technology, ensuring that
digital advances benefit all language communities equitably.

This course has opened doors to new ways of thinking about and engaging
with complex problems in my field. Rather than viewing research
methodology as a restrictive set of rules, I now see it as an enabling
framework that enhances the rigor, credibility, and impact of my work.
With these tools at my disposal, I am better equipped to make meaningful
contributions to my field while addressing real-world challenges that
matter to the communities I hope to serve.

\newpage

\section*{References}\label{references}
\addcontentsline{toc}{section}{References}

Joshi, P., Santy, S., Budhiraja, A., Bali, K., \& Choudhury, M. (2020).
The state and fate of linguistic diversity and inclusion in the NLP
world. In \emph{Proceedings of the 58th Annual Meeting of the
Association for Computational Linguistics} (pp.~6282--6293).

Marcus, G., \& Davis, E. (2020). \emph{Rebooting AI: Building artificial
intelligence we can trust}. Vintage.

Nekoto, W., Marivate, V., Matsila, T., Fasubaa, T., Kolawole, T.,
Fagbohungbe, T., \ldots{} \& Bashir, A. (2020). Participatory research
for low-resourced machine translation: A case study in African
languages. In \emph{Findings of the Association for Computational
Linguistics: EMNLP 2020} (pp.~2144--2160).




\end{document}
